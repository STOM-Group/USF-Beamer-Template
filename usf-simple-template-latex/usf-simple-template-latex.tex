% University of South Florida College of Engineering Beamer Template
% Garrick Aden-Buie
% Send comments/beers to: gadenbuie@mail.usf.edu

% Example presentation:
% http://garrickadenbuie.com/wp-content/uploads/2013/09/usf-beamer-example.pdf

% IMPORTANT: 
% If you want to use the USF logo, make sure you
% download the USF College of Engineering logo from:
% http://eng.usf.edu/about/news.asp
%
% All colors are USF-approved:
% http://usfweb4.usf.edu/ucm/marketing/colors.aspx

% CREDITS:
% Beamer framework modified from:
%  http://jeromyanglim.blogspot.com.au/2010/08/simple-beamer-template-for-getting.html
%
% Beamer template modified from example at:
% http://cameron.bracken.bz/beamer-template
%
% Conference Presentation Structure from example at:
% https://github.com/aecio/beamer-theme

% A great beamer intro/reference, A Beamer Quickstart:
% http://www.math.umbc.edu/~rouben/beamer/


%%%%%%%%%%%%%%%%%%%%%%%%%%%%%%%%%%%%%%
% 1. Presentation header file
%%%%%%%%%%%%%%%%%%%%%%%%%%%%%%%%%%%%%%
\documentclass[xcolor=x11names, compress]{beamer}

% If presenting on widescreen, use below:
%\usepackage[orientation=landscape,size=custom,width=16,height=9,scale=0.5,debug]{beamerposter}

\usepackage{graphicx}
\usepackage{amssymb,amsmath}
\usepackage{biblatex}
\usepackage{listings}
\usepackage{fancyvrb}
\usepackage{longtable}
\usepackage{url}

%%%%%%%%%%%%%%%%%%%%%%%%%%%%%%%%%%%%%%
% 2. Beamer Layout
%%%%%%%%%%%%%%%%%%%%%%%%%%%%%%%%%%%%%%

%%% USF Colors %%%
\definecolor{USFgreen}{RGB}{0,103,71}   		  %USF Official Dark Green
\definecolor{USFgold}{RGB}{207,196,147}			  %USF Official Gold
\definecolor{USFseaglass}{RGB}{128,176,166}		%USF Subdued Green/blue
\definecolor{USFtealgreen}{RGB}{0,147,116}    %USF Lighter teal green
\definecolor{USFsilvergray}{RGB}{126,150,160}	%USF Medium gray
\definecolor{USFslate}{RGB}{70,96,105}			  %USF Dark blue/gray
\definecolor{USFrhubarb}{RGB}{187,24,71}		  %USF brighter red
\definecolor{USFsky}{RGB}{41,175,206}			    %USF lighter blue
\definecolor{USFstorm}{RGB}{0,100,132}			  %USF Darker blue
\definecolor{USFpitaya}{RGB}{214,0,128}			  %USF classy pink
\definecolor{USFespresso}{RGB}{116,59,25}		  %USF brown
\definecolor{USFgrape}{RGB}{111,38,135}			  %USF purple
\definecolor{gDarkBlue}{RGB}{49,79,103}			  %my own Dark Blue color

%\usefonttheme[onlysmall]{structurebold}
\usepackage{arev}
\usepackage{arevmath}
\usepackage[T1]{fontenc}


% Palentino also looks great (but less readable for slide)
% Comment out above and use below if you want palatino
%\usefonttheme{serif}
%\usepackage{palatino}

\useoutertheme[subsection=false,shadow]{miniframes}
\useinnertheme{default}
\setbeamertemplate{mini frames}[default]

\setbeamerfont{title like}{series=\bfseries}
%\setbeamerfont{frametitle}{shape=\itshape}
\setbeamertemplate{navigation symbols}{}

\setbeamercolor*{normal text}{fg=black,bg=white} 
\setbeamercolor*{alerted text}{fg=USFrhubarb} 
\setbeamercolor*{example text}{fg=black} 
\setbeamercolor*{structure}{fg=black} 
\setbeamercolor{frametitle}{fg=USFrhubarb}

% head/foot colors
\setbeamercolor*{lower separation line head}{bg=USFgold} 
\setbeamercolor{footlinecolor}{bg=USFslate!20,fg=USFslate}
\setbeamercolor{section in head/foot}{bg=USFgreen, fg=USFgold}

% Block colors
\setbeamercolor{block title example}{bg=gDarkBlue, fg=white}
\setbeamercolor{block body example}{bg=gDarkBlue!20, fg=black}
\setbeamercolor{block title}{bg=USFgreen, fg=white}
\setbeamercolor{block body}{bg=USFgreen!10, fg=black}

\setbeamertemplate{footline}{%
\begin{beamercolorbox}{footlinecolor}
	\vskip2pt%
	\makebox[0pt][l]{\,\insertshortauthor}%
	\hspace*{\fill}\insertshorttitle\hspace*{\fill}%
	\llap{\insertpagenumber\,/\,\insertpresentationendpage\,}
%	\vskip2pt~ \insertshortauthor\hfill\insertshorttitle\hfill\insertpagenumber{} %
%   of \insertpresentationendpage{} ~\vskip2pt
\end{beamercolorbox}
}

\setbeamercolor*{palette tertiary}{fg=black,bg=black!10} 
\setbeamercolor*{palette quaternary}{fg=black,bg=black!10} 

\logo{%
	\vspace{-5pt}%
	\includegraphics[width=3cm,height=3cm,keepaspectratio]{usf-eng-horizontal.jpg}%
	%\hspace*{0.925\textwidth}
}

\renewcommand{\(}{\begin{columns}}
\renewcommand{\)}{\end{columns}}
\newcommand{\<}[1]{\begin{column}{#1}}
\renewcommand{\>}{\end{column}}

% The following adds footnote references
\usepackage[absolute,overlay]{textpos}
\newenvironment{reference}[2]{%
  \begin{textblock*}{0.8\textwidth}(#1,#2)
      \tiny\it\bgroup\color{USFslate}}{\egroup\end{textblock*}}

%%%%%%%%%%%%%%%%%%%%%%%%%%%%%%%%%%%%%%
% 3. Main content preamble and opening slides
%%%%%%%%%%%%%%%%%%%%%%%%%%%%%%%%%%%%%%

\title[Short Paper Title] % (optional, use only with long paper titles)
{Title As It Is In the Proceedings}

\subtitle
{Include Only If Paper Has a Subtitle}

\author[Author, Another] % (optional, use only with lots of authors)
{F.~Author\inst{1} \and S.~Another\inst{2}}
% - Give the names in the same order as the appear in the paper.
% - Use the \inst{?} command only if the authors have different
%   affiliation.

%\institute[Universities of Somewhere and Elsewhere] % (optional, but mostly needed)
%{
%  \inst{1}%
%  Department of Computer Science\\
%  University of Somewhere
%  \and
%  \inst{2}%
%  Department of Theoretical Philosophy\\
%  University of Elsewhere}

\institute{
	Industrial and Management Sciences Engineering\\
	~\\
	\includegraphics[height=1cm,keepaspectratio]{usf-eng-horizontal.jpg}}
  
% - Use the \inst command only if there are several affiliations.
% - Keep it simple, no one is interested in your street address.

\date[CFP 2003] % (optional, should be abbreviation of conference name)
{Conference on Fabulous Presentations, 2003}
% - Either use conference name or its abbreviation.
% - Not really informative to the audience, more for people (including
%   yourself) who are reading the slides online

\subject{Theoretical Computer Science}
% This is only inserted into the PDF information catalog. Can be left
% out. 



% If you have a file called "university-logo-filename.xxx", where xxx
% is a graphic format that can be processed by latex or pdflatex,
% resp., then you can add a logo as follows:

% \pgfdeclareimage[height=0.5cm]{university-logo}{university-logo-filename}
% \logo{\pgfuseimage{university-logo}}



% Delete this, if you do not want the table of contents to pop up at
% the beginning of each subsection:
\AtBeginSubsection[]
{
  \begin{frame}<beamer>{Outline}
    \tableofcontents[currentsection,currentsubsection]
  \end{frame}
}


% If you wish to uncover everything in a step-wise fashion, uncomment
% the following command: 

%\beamerdefaultoverlayspecification{<+->}


\begin{document}

\frame[plain]{
  \titlepage
}

\begin{frame}{Outline}
  \tableofcontents
  % You might wish to add the option [pausesections]
\end{frame}


% Structuring a talk is a difficult task and the following structure
% may not be suitable. Here are some rules that apply for this
% solution: 

% - Exactly two or three sections (other than the summary).
% - At *most* three subsections per section.
% - Talk about 30s to 2min per frame. So there should be between about
%   15 and 30 frames, all told.

% - A conference audience is likely to know very little of what you
%   are going to talk about. So *simplify*!
% - In a 20min talk, getting the main ideas across is hard
%   enough. Leave out details, even if it means being less precise than
%   you think necessary.
% - If you omit details that are vital to the proof/implementation,
%   just say so once. Everybody will be happy with that.

\section{Motivation}

\subsection{The Basic Problem That We Studied}

\begin{frame}{Make Titles Informative. Use Uppercase Letters.}
  % - A title should summarize the slide in an understandable fashion
  %   for anyone how does not follow everything on the slide itself.

  \begin{itemize}
  \item
    Use \texttt{itemize} a lot.
  \item
    Use very short sentences or short phrases.
  \end{itemize}
\end{frame}

\begin{frame}{Make Titles Informative.}

  You can create overlays\dots
  \begin{itemize}
  \item using the \texttt{pause} command:
    \begin{itemize}
    \item
      First item.
      \pause
    \item    
      Second item.
    \end{itemize}
  \item
    using overlay specifications:
    \begin{itemize}
    \item<3->
      First item.
    \item<4->
      Second item.
    \end{itemize}
  \item
    using the general \texttt{uncover} command:
    \begin{itemize}
      \uncover<5->{\item
        First item.}
      \uncover<6->{\item
        Second item.}
    \end{itemize}
  \end{itemize}
\end{frame}


\subsection{Previous Work}

\begin{frame}{Make Titles Informative.}
Previous research includes:
	\begin{itemize}	
	\item<1-> Research area 1
	\item<2-> Research area 2
	\item<3-> Research area 3
	\end{itemize}
\end{frame}

\begin{frame}{Make Titles Informative.}
This research builds on work in the areas of:
	\begin{itemize}	
	\item<1-> Research area 1
	\item<2-> Research area 2
	\item<3-> Research area 3
	\end{itemize}
\end{frame}



\section{Our Results/Contribution}

\subsection{Main Results}

\begin{frame}{Make Titles Informative.}
Main results 1.
\end{frame}

\begin{frame}{Make Titles Informative.}
Main results 2.
\end{frame}

\begin{frame}{Make Titles Informative.}
Main results 3.
\end{frame}


\subsection{Basic Ideas for Proofs/Implementation}

\begin{frame}{Make Titles Informative.}
\begin{theorem}
Theorem 1. Let $$\bar{x} = \frac{1}{N} \sum_{N} x_i$$
\end{theorem}
\end{frame}

\begin{frame}{Make Titles Informative.}
\begin{example}
This is an example of something happening.
\end{example}
\end{frame}

\begin{frame}{Make Titles Informative.}
Last proof idea/implementation.
\end{frame}



\section*{Summary}

\begin{frame}{Summary}

  % Keep the summary *very short*.
  \begin{itemize}
  \item
    The \alert{first main message} of your talk in one or two lines.
  \item
    The \alert{second main message} of your talk in one or two lines.
  \item
    Perhaps a \alert{third message}, but not more than that.
  \end{itemize}
  
  % The following outlook is optional.
  \vskip0pt plus.5fill
  \begin{itemize}
  \item
    Outlook
    \begin{itemize}
    \item
      Something you haven't solved.
    \item
      Something else you haven't solved.
    \end{itemize}
  \end{itemize}
\end{frame}



% All of the following is optional and typically not needed. 
\appendix
\section<presentation>*{\appendixname}
\subsection<presentation>*{For Further Reading}

\begin{frame}[allowframebreaks]
  \frametitle<presentation>{For Further Reading}
    
  \begin{thebibliography}{10}
    
  \beamertemplatebookbibitems
  % Start with overview books.

  \bibitem{Author1990}
    A.~Author.
    \newblock {\em Handbook of Everything}.
    \newblock Some Press, 1990.
 
    
  \beamertemplatearticlebibitems
  % Followed by interesting articles. Keep the list short. 

  \bibitem{Someone2000}
    S.~Someone.
    \newblock On this and that.
    \newblock {\em Journal of This and That}, 2(1):50--100,
    2000.
  \end{thebibliography}
\end{frame}

\end{document}




%%%%%%%%%%%%%%%%%%%%%%%%%%%%%%%%%%%%%%
% 4. Templates for individual slide types
%    Build slide using these!
%%%%%%%%%%%%%%%%%%%%%%%%%%%%%%%%%%%%%%

% Add a reference footnote to a particular slide
% Normal Screen Size Only (don't bother if widescreen)
   \begin{reference}{2mm}{86.5mm}
       Full reference here.
   \end{reference} 

% Slide at the beginning of a section
\begin{frame}{Outline}
\tableofcontents[currentsection]
\end{frame}


% Simple slide
\begin{frame}{}
\begin{itemize}
\item
\end{itemize}
\end{frame}


% Slide with a block
\begin{frame}{}
\begin{block}{}
\begin{itemize}
\item
\end{itemize}
\end{block}
\end{frame}


% Slide with example (different colour to Block)
\begin{frame}{}
\begin{example}
\begin{itemize}
\item
\end{itemize}
\end{example}
\end{frame}


% Slide with two columns
% (0.5 can be adjusted but should generally sum to 1.0)
\begin{frame}{}
\begin{columns}
\begin{column}{0.5\textwidth}
\end{column}
\begin{column}{0.5\textwidth}
\end{column}
\end{columns}
\end{frame}


% Verbatim
\begin{frame}[fragile]{}
\begin{verbatim}
% code
\end{verbatim}
\end{frame}

% slide with title and figure
\begin{frame}{}
\begin{center}
\includegraphics[width=11cm]{files/figure}
%\caption
\end{center}
\end{frame}


% Full screen image
\mode<all>
{\usebackgroundtemplate{\includegraphics[width=\paperwidth]
{files/figure}} %replace files/figure with file path and name
\begin{frame}[plain]
\end{frame}}
\mode<all>{\usebackgroundtemplate{}}
\mode*


% Basic centred table with alternating colour rows
% \usepackage{xcolor}
% \documentclass[xcolor=pdftex,dvipsnames, table]{beamer}
% adapted from http://www.tug.org/TUGboat/Articles/tb26-1/mertz.pdf
\begin{frame}{}
\begin{center}
\rowcolors{1}{\RoyalBlue!20}{\RoyalBlue!5}
\begin{tabular}{lll}\hline
A & B & C \\
\end{tabular}
\end{center}
\end{frame}


% Frame with incrementally displayed dot points with an alert
\begin{frame}{}
\begin{itemize}[<+-| alert@+>]
\item
\end{itemize}
\end{frame}
