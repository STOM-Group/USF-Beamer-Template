% !TEX TS-program = xelatex
% !TEX encoding = UTF-8 Unicode

\documentclass[10pt]{beamer}

\usetheme{Tampa}
%\usecolortheme{Delta}
% color theme options: USFyellow, USFgold, Delta  (default: USFyellow)

% Other Beamer color themes:  \usecolortheme{<color theme name>}
% albatross, beaver, beetle, crane, default, dolphin, dove, fly, lily, orchid, rose, seagull, seahorse, sidebartab, structure, whale, wolverine



% Some font recommendation, compile with xelatex
% Get Fira Sans fonts:   https://github.com/mozilla/Fira

% Font setting 
\usepackage{fontspec}
\setsansfont[ItalicFont={Fira Sans Light Italic},%
             BoldFont={Fira Sans},%
             BoldItalicFont={Fira Sans Italic}]%
            {Fira Sans Light}%
\setmonofont[BoldFont={Fira Mono Medium}]{Fira Mono}%

% or simply use metropolis for Fira Sans
% metropolis theme:  https://github.com/matze/mtheme
%\usefonttheme{metropolis} 




%%%%%%%%%%%%%%%%%%%%%%%%%%%%%%%%%%%%%%%%%%%%%%%%%%%%%%%%%%%%%%%%%%%%%
% Custom Packages and Macros
%%%%%%%%%%%%%%%%%%%%%%%%%%%%%%%%%%%%%%%%%%%%%%%%%%%%%%%%%%%%%%%%%%%%%

\usepackage{mwe}
\usepackage{amsmath}
\usepackage{amsfonts}
\usepackage{amssymb}
\usepackage{graphicx}
\usepackage{multicol}
\usepackage{multirow}
\usepackage{color}
\usepackage{natbib}
\usepackage{booktabs}
\usepackage{bbold}
\usepackage{tikz}
\usetikzlibrary{positioning,shapes,shadows,arrows}

\usepackage{etoolbox}
\makeatletter
\def\do#1{\@namedef{#1c}{\ensuremath{\mathcal{#1}}}}
\docsvlist{A,B,C,D,E,F,G,H,I,J,K,L,M,N,O,P,Q,R,S,T,U,V,W,X,Y,Z}
\makeatother

\makeatletter
\def\do#1{\@namedef{#1b}{\ensuremath{\mathbb{#1}}}}
\docsvlist{A,B,C,D,E,F,G,H,I,J,K,L,M,N,O,P,Q,R,S,T,U,V,W,X,Y,Z}
\makeatother

\newcommand{\red}[1]{{\color{red}#1}}
\newcommand{\blue}[1]{{\color{blue}#1}}
\newcommand{\green}[1]{{\color{green!60!white}#1}}

\newcommand{\email}[1]{{\href{mailto:#1}{\nolinkurl{#1}}}}

\usepackage{bm}
\renewcommand{\vec}[1]{\boldsymbol{\mathbf{#1}}}
\newcommand{\mat}[1]{\vec{#1}}
\newcommand{\set}[1]{\boldsymbol{\mathbf{#1}}}
\newcommand{\optset}[1]{\mathbb{#1}}

%\renewcommand{\bar}[1]{\mkern 1.5mu\overline{\mkern-1.5mu#1\mkern-1.5mu}\mkern 1.5mu}
%\renewcommand{\hat}{\widehat}
%\renewcommand{\tilde}{\widetilde}

\newcommand{\ds}{\mathop{}\!\mathrm{d}s}
\newcommand{\dx}{\mathop{}\!\mathrm{d}x}
\newcommand{\dy}{\mathop{}\!\mathrm{d}y}
\newcommand{\dz}{\mathop{}\!\mathrm{d}z}
\newcommand{\dt}{\mathop{}\!\mathrm{d}t}
\DeclareMathOperator{\ord}{ord}

\DeclareMathOperator*{\minimize}{minimize}
\DeclareMathOperator*{\maximize}{maximize}
\DeclareMathOperator*{\argmax}{\arg \max}
\DeclareMathOperator*{\argmin}{\arg \min}
\DeclareMathOperator*{\conv}{conv}


%%%%%%%%%%%%%%%%%%%%%%%%%%%%%%%%%%%%%%%%%%%%%%%%%%%%%%%%%%%%%%%%%%%%%


\everymath{\displaystyle}










\title{Presentation Slide Template}
\author[C Kwon]{Someone Else\inst{1} \and Changhyun Kwon\inst{2}}
\institute[USF]{
  \inst{1}Department of XXX\\University of YYYY
  \and
  \inst{2}Department of Industrial and Management Systems Engineering\\University of South Florida}
\date[CONFERENCE 2020]{Very Important Conference 2020}




\begin{document}






% Title page
\frame[plain]{\titlepage}





\section[INTRO]{Introduction}

\begin{frame}

   \begin{block}{This is a Block}
      This is important information
   \end{block}

\pause

   \begin{alertblock}{This is an Alert block}
   This is an important alert
   \end{alertblock}

\pause

   \begin{exampleblock}{This is an Example block}
   This is an example 
   \end{exampleblock}

\end{frame}

%%%%%%%%%%%%%%%%%%%%%%%%%%%%%%%%%%%

\begin{frame}
\frametitle{SLIDE TITLE...}

Hello $\int_0^x y dy$ \alert<2>{ddd}
\begin{itemize}
\item Queen $x^2 + y^2 = z^2$
\item Hello World $\mat{A}^\top \vec{x} = \vec{0}$
\item Good Morning $a\in\Ac$, $n\in\Nc$, $\Bb$, $\Eb$
\item Good Night $\sum_{(i,j)\in\Ac} x_{ij} y_{ij} \leq 1$ 
\end{itemize}

\begin{itemize}
\item<1-> Hello \alert{World}
\item<2-> Good \alert{Morning}
\item<3-> Good \alert{Night}
\end{itemize}

\end{frame}



\begin{frame}
\frametitle{LISTS...}
\texttt{itemize:}
\begin{itemize}
\item Hello \alert{World}
\item Good \alert{Morning}

  \begin{itemize}
  \item Hello \alert{World}
  \item Good \alert{Morning}
  
    \begin{itemize}
    \item Hello \alert{World}
    \item Good \alert{Morning}
    \end{itemize}
  
  \end{itemize}

\end{itemize}


\texttt{enumerate:}
\begin{enumerate}
\item Hello World
\item Good Morning
\end{enumerate}

\texttt{description:}
\begin{description}
\item [Hello] World
\item [Good] Morning
\end{description}

\end{frame}





\begin{frame}
\frametitle{THEOREMS}

\begin{theorem}
Hello theorem
\[
	\sum_{i=1}^n x_i = 1
\]
\end{theorem}
\end{frame}

\begin{frame}
\frametitle{THEOREMS}

\begin{theorem}
Hello theorem, you can put ``pause'' anywhere
\pause
\[
	\sum_{i=1}^n x_i = 1
\]
\end{theorem}
\end{frame}


%%%%%%%%%%%%%%%%%%%%%%%%%%%%%%%%%%%
%%%%%%%%%%%%%%%%%%%%%%%%%%%%%%%%%%%

\section[FIGURES]{Figures}

%%%%%%%%%%%%%%%%%%%%%%%%%%%%%%%%%%%
%%%%%%%%%%%%%%%%%%%%%%%%%%%%%%%%%%%


\begin{frame}
\frametitle{IMAGE OVERLAY}
\begin{center}
\includegraphics<1>[width=0.7\textwidth]{image-a}
\includegraphics<2>[width=0.7\textwidth]{image-b}
\includegraphics<3>[width=0.7\textwidth]{image-c}
\end{center}
\end{frame}







%%%%%%%%%%%%%%%%%%%%%%%%%%%%%%%%%%%
%%%%%%%%%%%%%%%%%%%%%%%%%%%%%%%%%%%

\section[CONCLUSION]{Conclusion}

%%%%%%%%%%%%%%%%%%%%%%%%%%%%%%%%%%%
%%%%%%%%%%%%%%%%%%%%%%%%%%%%%%%%%%%

\begin{frame}
\frametitle{GOODBYE}

Good bye world. \email{chkwon@usf.edu}

\end{frame}





\end{document}